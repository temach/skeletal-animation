\documentclass[twoside]{article}

% Packages required by doxygen
\usepackage{fixltx2e}
\usepackage{calc}
\usepackage{doxygen}
\usepackage[export]{adjustbox} % also loads graphicx
\usepackage{graphicx}
\usepackage[utf8]{inputenc}
\usepackage{makeidx}
\usepackage{multicol}
\usepackage{multirow}
\PassOptionsToPackage{warn}{textcomp}
\usepackage{textcomp}
\usepackage[nointegrals]{wasysym}
\usepackage[table]{xcolor}

% NLS support packages
\usepackage[T2A]{fontenc}
\usepackage[russian]{babel}

% Font selection
\usepackage[T1]{fontenc}
\usepackage[scaled=.90]{helvet}
\usepackage{courier}
\usepackage{amssymb}
\usepackage{sectsty}
\renewcommand{\familydefault}{\sfdefault}
\allsectionsfont{%
  \fontseries{bc}\selectfont%
  \color{darkgray}%
}
\renewcommand{\DoxyLabelFont}{%
  \fontseries{bc}\selectfont%
  \color{darkgray}%
}
\newcommand{\+}{\discretionary{\mbox{\scriptsize$\hookleftarrow$}}{}{}}

% Page & text layout
\usepackage{geometry}
\geometry{%
  a4paper,%
  top=2.5cm,%
  bottom=2.5cm,%
  left=2.5cm,%
  right=2.5cm%
}
\tolerance=750
\hfuzz=15pt
\hbadness=750
\setlength{\emergencystretch}{15pt}
\setlength{\parindent}{0cm}
\setlength{\parskip}{0.2cm}
\makeatletter
\renewcommand{\paragraph}{%
  \@startsection{paragraph}{4}{0ex}{-1.0ex}{1.0ex}{%
    \normalfont\normalsize\bfseries\SS@parafont%
  }%
}
\renewcommand{\subparagraph}{%
  \@startsection{subparagraph}{5}{0ex}{-1.0ex}{1.0ex}{%
    \normalfont\normalsize\bfseries\SS@subparafont%
  }%
}
\makeatother

% Headers & footers
\usepackage{fancyhdr}
\pagestyle{fancyplain}
\fancyhead[LE]{\fancyplain{}{\bfseries\thepage}}
\fancyhead[CE]{\fancyplain{}{}}
\fancyhead[RE]{\fancyplain{}{\bfseries\leftmark}}
\fancyhead[LO]{\fancyplain{}{\bfseries\rightmark}}
\fancyhead[CO]{\fancyplain{}{}}
\fancyhead[RO]{\fancyplain{}{\bfseries\thepage}}
\fancyfoot[LE]{\fancyplain{}{}}
\fancyfoot[CE]{\fancyplain{}{}}
\fancyfoot[RE]{\fancyplain{}{\bfseries\scriptsize Документация по КДЗ модуль 3. }}
\fancyfoot[LO]{\fancyplain{}{\bfseries\scriptsize Документация по КДЗ модуль 3. }}
\fancyfoot[CO]{\fancyplain{}{}}
\fancyfoot[RO]{\fancyplain{}{}}
\renewcommand{\footrulewidth}{0.4pt}
\renewcommand{\sectionmark}[1]{%
  \markright{\thesection\ #1}%
}

% Indices & bibliography
\usepackage{natbib}
\usepackage[titles]{tocloft}
\setcounter{tocdepth}{3}
\setcounter{secnumdepth}{5}
\makeindex

% Custom commands
\newcommand{\clearemptydoublepage}{%
  \newpage{\pagestyle{empty}\cleardoublepage}%
}

% Custom packages
\usepackage{pdfpages}


%===== C O N T E N T S =====

\begin{document}

% Titlepage & ToC
\pagenumbering{roman}
\begin{titlepage}
\begin{center}
\vspace*{1cm}
{\large НАЦИОНАЛЬНЫЙ ИССЛЕДОВАТЕЛЬСКИЙ УНИВЕРСИТЕТ \\
«ВЫСШАЯ ШКОЛА ЭКОНОМИКИ» }\\
\vspace*{0.5cm}
{\large Факультет компьютерных наук }\\
\vspace*{0.5cm}
{\small Департамент программнoй инженерии \\
}
\vfill % заполняет длину страницы вертикально
{\large\textbf{
Контрольное домашнее задание \\
по дисциплине\\
«Программирование» \\
}}
\bigskip
{\large Тема работы: Обработка данных из файла }\\
\vfill
\begin{flushright}
Выполнил студент группы БПИ 151 \\
Абрамов А.M. \\
Преподаватель: Подбельский Вадим Валериевич \\
\end{flushright}
\vfill
Москва \number\year \\
Модуль 3
\end{center}
\end{titlepage}

\tableofcontents
\pagenumbering{arabic}

% --- add my custom headers ---


%--- Begin generated contents ---
\section{Титульная страница}
\label{index}This program is a demo of skeletal animation in C\#. Feed it collada files with animation and bone data.

In blender spread the mesh on X-\/Y plane (so Z is normal vector) Also make it in the X$>$0, Y$>$0 quadrant. When importing a collada file remember to modify it by hand. You should change Z\+\_\+\+UP to Y\+\_\+\+UP. (I think this is only necessary while we work with Grpahics object, when we go to opengl this should fix itself) 
\section{Пространства имен}
\subsection{Пространство имен Win\+Form\+Animation2D}
\label{namespace_win_form_animation2_d}\index{Win\+Form\+Animation2D@{Win\+Form\+Animation2D}}
\subsubsection*{Классы}
\begin{DoxyCompactItemize}
\item 
class {\bf Action\+State}
\item 
class {\bf Camera\+Device}
\item 
class {\bf Draw\+Config}
\item 
class {\bf Entity}
\item 
class {\bf Geometry}
\item 
interface {\bf I\+Transform\+State}
\item 
class {\bf Mesh\+Draw}
\item 
class {\bf Mouse\+State}
\item 
class {\bf Renderer}
\end{DoxyCompactItemize}

\section{Классы}
\subsection{Класс Action\+State}
\label{class_win_form_animation2_d_1_1_action_state}\index{Action\+State@{Action\+State}}


Базовые классы\+:Base\+For\+Event\+Driven.



\subsubsection{Подробное описание}
This class knows what argumets to pass to Node\+Interpolator 


\subsection{Класс Camera\+Device}
\label{class_win_form_animation2_d_1_1_camera_device}\index{Camera\+Device@{Camera\+Device}}
\subsubsection*{Открытые члены}
\begin{DoxyCompactItemize}
\item 
Vector3 {\bf Convert\+Screen2\+World\+Coordinates} (Point screen\+\_\+coords)\label{class_win_form_animation2_d_1_1_camera_device_ac7ad419940551e2a63d36549481ff424}

\item 
Matrix4 {\bf Matrix\+To\+Open\+GL} ()\label{class_win_form_animation2_d_1_1_camera_device_a708b9cba847463c03fc39aa0e6a8e51b}

\end{DoxyCompactItemize}


\subsubsection{Подробное описание}
Maintains camera abstraction. 


\subsection{Класс Draw\+Config}
\label{class_win_form_animation2_d_1_1_draw_config}\index{Draw\+Config@{Draw\+Config}}
\subsubsection*{Открытые атрибуты}
\begin{DoxyCompactItemize}
\item 
bool {\bf Enable\+Texture2D} = false
\end{DoxyCompactItemize}


\subsubsection{Подробное описание}
This class will be passed to the \doxyref{Entity}{стр.}{class_win_form_animation2_d_1_1_entity}\textquotesingle{}s Get\+Settings() so it make the scene look best. 

\subsubsection{Данные класса}
\index{Win\+Form\+Animation2\+D\+::\+Draw\+Config@{Win\+Form\+Animation2\+D\+::\+Draw\+Config}!Enable\+Texture2D@{Enable\+Texture2D}}
\index{Enable\+Texture2D@{Enable\+Texture2D}!Win\+Form\+Animation2\+D\+::\+Draw\+Config@{Win\+Form\+Animation2\+D\+::\+Draw\+Config}}
\paragraph[{Enable\+Texture2D}]{\setlength{\rightskip}{0pt plus 5cm}bool Enable\+Texture2D = false}\label{class_win_form_animation2_d_1_1_draw_config_ab0185dd3fb4ebf71f25125334556bf8f}
Open\+GL settings here is a template\+: public bool \+\_\+enable = false; 
\subsection{Класс Entity}
\label{class_win_form_animation2_d_1_1_entity}\index{Entity@{Entity}}
\subsubsection*{Открытые члены}
\begin{DoxyCompactItemize}
\item 
void {\bf Recursive\+Calculate\+Vertex\+Transform} (Node nd, Matrix4x4 current)
\item 
void {\bf Render\+Model} ({\bf Draw\+Config} settings)\label{class_win_form_animation2_d_1_1_entity_ae7e350876ed1f078eb5a780c03b473a4}

\end{DoxyCompactItemize}


\subsubsection{Подробное описание}
Represents the currently loaded object. We will have lots of these. 



\subsubsection{Методы}
\index{Win\+Form\+Animation2\+D\+::\+Entity@{Win\+Form\+Animation2\+D\+::\+Entity}!Recursive\+Calculate\+Vertex\+Transform@{Recursive\+Calculate\+Vertex\+Transform}}
\index{Recursive\+Calculate\+Vertex\+Transform@{Recursive\+Calculate\+Vertex\+Transform}!Win\+Form\+Animation2\+D\+::\+Entity@{Win\+Form\+Animation2\+D\+::\+Entity}}
\paragraph[{Recursive\+Calculate\+Vertex\+Transform(\+Node nd, Matrix4x4 current)}]{\setlength{\rightskip}{0pt plus 5cm}void Recursive\+Calculate\+Vertex\+Transform (
\begin{DoxyParamCaption}
\item[{Node}]{nd, }
\item[{Matrix4x4}]{current}
\end{DoxyParamCaption}
)}\label{class_win_form_animation2_d_1_1_entity_a4f652bc4baf5a8f594384aab653088bf}
bind tells the original delta in global coord, so we can find current delta 
\subsection{Класс Geometry}
\label{class_win_form_animation2_d_1_1_geometry}\index{Geometry@{Geometry}}
\subsubsection*{Открытые члены}
\begin{DoxyCompactItemize}
\item 
{\bf Geometry} (I\+List$<$ Mesh $>$ scene\+\_\+meshes, Node nd, Bone\+Node armature)\label{class_win_form_animation2_d_1_1_geometry_abd05402398b233a02784515ecd0d329e}

\end{DoxyCompactItemize}


\subsubsection{Подробное описание}
Stores info on extra geometry of the entity. 
\subsection{Интерфейс I\+Transform\+State}
\label{interface_win_form_animation2_d_1_1_i_transform_state}\index{I\+Transform\+State@{I\+Transform\+State}}


Производные классы\+:Camera\+Drawing2D и Camera\+Free\+Fly3D.



\subsubsection{Подробное описание}
Implement this when class allows local matrix transforms. (\doxyref{Entity}{стр.}{class_win_form_animation2_d_1_1_entity}, Camera) 


\subsection{Класс Mesh\+Draw}
\label{class_win_form_animation2_d_1_1_mesh_draw}\index{Mesh\+Draw@{Mesh\+Draw}}
\subsubsection*{Открытые члены}
\begin{DoxyCompactItemize}
\item 
{\bf Mesh\+Draw} (Mesh mesh, I\+List$<$ Material $>$ materials)
\item 
void {\bf Render\+V\+BO} ()
\end{DoxyCompactItemize}


\subsubsection{Подробное описание}
Mesh rendering using V\+B\+Os. 

Based on {\tt http\+://www.\+opentk.\+com/files/\+T08\+\_\+\+V\+B\+O.\+cs} 

\subsubsection{Конструктор(ы)}
\index{Win\+Form\+Animation2\+D\+::\+Mesh\+Draw@{Win\+Form\+Animation2\+D\+::\+Mesh\+Draw}!Mesh\+Draw@{Mesh\+Draw}}
\index{Mesh\+Draw@{Mesh\+Draw}!Win\+Form\+Animation2\+D\+::\+Mesh\+Draw@{Win\+Form\+Animation2\+D\+::\+Mesh\+Draw}}
\paragraph[{Mesh\+Draw(\+Mesh mesh, I\+List$<$ Material $>$ materials)}]{\setlength{\rightskip}{0pt plus 5cm}{\bf Mesh\+Draw} (
\begin{DoxyParamCaption}
\item[{Mesh}]{mesh, }
\item[{I\+List$<$ Material $>$}]{materials}
\end{DoxyParamCaption}
)}\label{class_win_form_animation2_d_1_1_mesh_draw_a70daeba67e8888df740583158e434490}


Uploads the data to the G\+PU. 



\subsubsection{Методы}
\index{Win\+Form\+Animation2\+D\+::\+Mesh\+Draw@{Win\+Form\+Animation2\+D\+::\+Mesh\+Draw}!Render\+V\+BO@{Render\+V\+BO}}
\index{Render\+V\+BO@{Render\+V\+BO}!Win\+Form\+Animation2\+D\+::\+Mesh\+Draw@{Win\+Form\+Animation2\+D\+::\+Mesh\+Draw}}
\paragraph[{Render\+V\+B\+O()}]{\setlength{\rightskip}{0pt plus 5cm}void Render\+V\+BO (
\begin{DoxyParamCaption}
{}
\end{DoxyParamCaption}
)}\label{class_win_form_animation2_d_1_1_mesh_draw_a7735f3e5b99af7ffb9e8343a36c3d688}


Render mesh from G\+PU memory. The pipeline is restored afterwards. 


\subsection{Класс Mouse\+State}
\label{class_win_form_animation2_d_1_1_mouse_state}\index{Mouse\+State@{Mouse\+State}}
\subsubsection*{Открытые члены}
\begin{DoxyCompactItemize}
\item 
void {\bf Record\+Mouse\+Click} (Mouse\+Event\+Args e)\label{class_win_form_animation2_d_1_1_mouse_state_a90501009d4cbf911a31d880523548f63}

\item 
void {\bf Record\+Mouse\+Move} (Mouse\+Event\+Args e)\label{class_win_form_animation2_d_1_1_mouse_state_afa4a1b4f3fc39ff61fd13d7ba3e5d628}

\end{DoxyCompactItemize}
\subsubsection*{Открытые атрибуты}
\begin{DoxyCompactItemize}
\item 
Point {\bf Click\+Pos}\label{class_win_form_animation2_d_1_1_mouse_state_aff1227a0577100876a98c04ad407d9f4}

\item 
Point {\bf Current\+Pos}\label{class_win_form_animation2_d_1_1_mouse_state_aa03fd24e8bc7ea3e7ed22a9c564cb650}

\item 
readonly int {\bf Horiz\+Hysteresis} = 4\label{class_win_form_animation2_d_1_1_mouse_state_a2564dfd90e7599c1f850ecc35b4243ea}

\item 
Open\+T\+K.\+Vector3 {\bf Inner\+World\+Click\+Pos}\label{class_win_form_animation2_d_1_1_mouse_state_a2ba32d9d742f09b887d7487ead69490c}

\item 
Open\+T\+K.\+Vector3 {\bf Inner\+World\+Pos}\label{class_win_form_animation2_d_1_1_mouse_state_ae4a531cfd584f7535a84d3491c0d233a}

\end{DoxyCompactItemize}


\subsubsection{Подробное описание}
Simple class to store mouse status data. Monitor mouse status (delta, position, click\+\_\+position, etc.) 


\subsection{Класс Renderer}
\label{class_win_form_animation2_d_1_1_renderer}\index{Renderer@{Renderer}}
\subsubsection*{Открытые члены}
\begin{DoxyCompactItemize}
\item 
void {\bf Draw\+Axis3D} ()
\end{DoxyCompactItemize}


\subsubsection{Подробное описание}
Class to control open\+GL settings and do the actual drawing. All open\+GL calls will be here. 



\subsubsection{Методы}
\index{Win\+Form\+Animation2\+D\+::\+Renderer@{Win\+Form\+Animation2\+D\+::\+Renderer}!Draw\+Axis3D@{Draw\+Axis3D}}
\index{Draw\+Axis3D@{Draw\+Axis3D}!Win\+Form\+Animation2\+D\+::\+Renderer@{Win\+Form\+Animation2\+D\+::\+Renderer}}
\paragraph[{Draw\+Axis3\+D()}]{\setlength{\rightskip}{0pt plus 5cm}void Draw\+Axis3D (
\begin{DoxyParamCaption}
{}
\end{DoxyParamCaption}
)}\label{class_win_form_animation2_d_1_1_renderer_a78178415342aec84aff33ff22fd711eb}


Important points to remember\+: Set normals. Must be clock wise vertex draw order The x-\/axis is accross the screen, so the Z-\/axis triangle must have component along X\+: +-\/1 since look at looks towards the center, we need to offset it a bit to see the Z axis. 


%--- End generated contents ---

% Index
\newpage
\phantomsection
\clearemptydoublepage
\addcontentsline{toc}{section}{Алфавитный указатель}
\printindex

\end{document}
