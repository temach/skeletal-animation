\documentclass[twoside]{article}

% Packages required by doxygen
\usepackage{fixltx2e}
\usepackage{calc}
\usepackage{doxygen}
\usepackage[export]{adjustbox} % also loads graphicx
\usepackage{graphicx}
\usepackage[utf8]{inputenc}
\usepackage{makeidx}
\usepackage{multicol}
\usepackage{multirow}
\PassOptionsToPackage{warn}{textcomp}
\usepackage{textcomp}
\usepackage[nointegrals]{wasysym}
\usepackage[table]{xcolor}

% NLS support packages
\usepackage[T2A]{fontenc}
\usepackage[russian]{babel}

% Font selection
\usepackage[T1]{fontenc}
\usepackage[scaled=.90]{helvet}
\usepackage{courier}
\usepackage{amssymb}
\usepackage{sectsty}
\renewcommand{\familydefault}{\sfdefault}
\allsectionsfont{%
  \fontseries{bc}\selectfont%
  \color{darkgray}%
}
\renewcommand{\DoxyLabelFont}{%
  \fontseries{bc}\selectfont%
  \color{darkgray}%
}
\newcommand{\+}{\discretionary{\mbox{\scriptsize$\hookleftarrow$}}{}{}}

% Page & text layout
\usepackage{geometry}
\geometry{%
  a4paper,%
  top=2.5cm,%
  bottom=2.5cm,%
  left=2.5cm,%
  right=2.5cm%
}
\tolerance=750
\hfuzz=15pt
\hbadness=750
\setlength{\emergencystretch}{15pt}
\setlength{\parindent}{0cm}
\setlength{\parskip}{0.2cm}
\makeatletter
\renewcommand{\paragraph}{%
  \@startsection{paragraph}{4}{0ex}{-1.0ex}{1.0ex}{%
    \normalfont\normalsize\bfseries\SS@parafont%
  }%
}
\renewcommand{\subparagraph}{%
  \@startsection{subparagraph}{5}{0ex}{-1.0ex}{1.0ex}{%
    \normalfont\normalsize\bfseries\SS@subparafont%
  }%
}
\makeatother

% Headers & footers
\usepackage{fancyhdr}
\pagestyle{fancyplain}
\fancyhead[LE]{\fancyplain{}{\bfseries\thepage}}
\fancyhead[CE]{\fancyplain{}{}}
\fancyhead[RE]{\fancyplain{}{\bfseries\leftmark}}
\fancyhead[LO]{\fancyplain{}{\bfseries\rightmark}}
\fancyhead[CO]{\fancyplain{}{}}
\fancyhead[RO]{\fancyplain{}{\bfseries\thepage}}
\fancyfoot[LE]{\fancyplain{}{}}
\fancyfoot[CE]{\fancyplain{}{}}
\fancyfoot[RE]{\fancyplain{}{\bfseries\scriptsize Документация по КДЗ модуль 3. }}
\fancyfoot[LO]{\fancyplain{}{\bfseries\scriptsize Документация по КДЗ модуль 3. }}
\fancyfoot[CO]{\fancyplain{}{}}
\fancyfoot[RO]{\fancyplain{}{}}
\renewcommand{\footrulewidth}{0.4pt}
\renewcommand{\sectionmark}[1]{%
  \markright{\thesection\ #1}%
}

% Indices & bibliography
\usepackage{natbib}
\usepackage[titles]{tocloft}
\setcounter{tocdepth}{3}
\setcounter{secnumdepth}{5}
\makeindex

% Custom commands
\newcommand{\clearemptydoublepage}{%
  \newpage{\pagestyle{empty}\cleardoublepage}%
}

% Custom packages
\usepackage{pdfpages}


%===== C O N T E N T S =====

\begin{document}

% Titlepage & ToC
\pagenumbering{roman}
\begin{titlepage}
\begin{center}
\vspace*{1cm}
{\large НАЦИОНАЛЬНЫЙ ИССЛЕДОВАТЕЛЬСКИЙ УНИВЕРСИТЕТ \\
«ВЫСШАЯ ШКОЛА ЭКОНОМИКИ» }\\
\vspace*{0.5cm}
{\large Факультет компьютерных наук }\\
\vspace*{0.5cm}
{\small Департамент программнoй инженерии \\
}
\vfill % заполняет длину страницы вертикально
{\large\textbf{
Контрольное домашнее задание \\
по дисциплине\\
«Программирование» \\
}}
\bigskip
{\large Тема работы: Обработка данных из файла }\\
\vfill
\begin{flushright}
Выполнил студент группы БПИ 151 \\
Абрамов А.M. \\
Преподаватель: Подбельский Вадим Валериевич \\
\end{flushright}
\vfill
Москва \number\year \\
Модуль 3
\end{center}
\end{titlepage}

\tableofcontents
\pagenumbering{arabic}

% --- add my custom headers ---


%--- Begin generated contents ---
\section{Титульная страница}
\label{index}This program is a demo of skeletal animation in C\#. Feed it collada files with animation and bone data.

In blender spread the mesh on X-\/Y plane (so Z is normal vector) Also make it in the X$>$0, Y$>$0 quadrant. When importing a collada file remember to modify it by hand. You should change Z\+\_\+\+UP to Y\+\_\+\+UP. (I think this is only necessary while we work with Grpahics object, when we go to opengl this should fix itself) 
\section{Пространства имен}
\subsection{Пространство имен Win\+Form\+Animation2D}
\label{namespace_win_form_animation2_d}\index{Win\+Form\+Animation2D@{Win\+Form\+Animation2D}}
\subsubsection*{Классы}
\begin{DoxyCompactItemize}
\item 
class {\bf Action\+State}
\item 
class {\bf Camera\+Device}
\item 
class {\bf Draw\+Config}
\item 
class {\bf Entity}
\item 
class {\bf Geometry}
\item 
interface {\bf I\+Transform\+State}
\item 
class {\bf Mesh\+Draw}
\item 
class {\bf Mouse\+State}
\item 
class {\bf Renderer}
\end{DoxyCompactItemize}

\section{Классы}
\input{class_win_form_animation2_d_1_1_action_state}
\input{class_win_form_animation2_d_1_1_camera_device}
\input{class_win_form_animation2_d_1_1_draw_config}
\input{class_win_form_animation2_d_1_1_entity}
\input{class_win_form_animation2_d_1_1_geometry}
\input{interface_win_form_animation2_d_1_1_i_transform_state}
\input{class_win_form_animation2_d_1_1_mesh_draw}
\input{class_win_form_animation2_d_1_1_mouse_state}
\input{class_win_form_animation2_d_1_1_renderer}
%--- End generated contents ---

% Index
\newpage
\phantomsection
\clearemptydoublepage
\addcontentsline{toc}{section}{Алфавитный указатель}
\printindex

\end{document}
